\chapter{Overview - State-of-the-art}
\label{chapter:sota}

concrete header for the given thesis's theme
    \cite{goodfellow2016deep}
    \cite{keras}
    \cite{liu2021machine}

\section{Background}
\label{sec:sota-background}

Don't write too long about well-known facts in books and text-books.


\section{Related Work (to your work)}
\label{sec:sota-related}

This part is important in determining the objectives of the Thesis as we will state in Section~\ref{sec:sota-summary} and in Chapter~\ref{chapter:objectives-and-methodology}.


\section{Summary and Starting Points}
\label{sec:sota-summary}

% \section{Starting Points}
% \label{sec:sota-starting}

Summary of Chapter~\ref{chapter:sota} ... several paragraphs... 1/2 to 1 page

Based on the study presented in Chapter~\ref{chapter:sota}, the following gaps / needs are identified with potential to improve:
\begin{itemize}
    \item question /  gap / need
    \item ...
    \item question /  gap / need
\end{itemize}


To answer these questions, the following starting points are proposed.
\begin{itemize}
    \item Starting point 1:
    \item Starting point 2:
    \item Starting point 3:
\end{itemize}

\newpage
\textbf{Priebežná správa DP1 musí obsahovať:}
\begin{enumerate}
    \item Prehľad podobných riešení alebo produktov, s ktorými sa študent oboznámil pri štúdiu danej problematiky.
    \item Špecifikáciu požiadaviek na vytvárané riešenie alebo produkt
            (Chapter~\ref{chapter:objectives-and-methodology}).
    \item Návrh
    \item Vlastne zhodnotenie práce za príslušné obdobie 
            (Chapter~\ref{chapter:Work-Schedule} - 
            Section~\ref{sec:Work-Schedule-1st-semester}).
\end{enumerate}
